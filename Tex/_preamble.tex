%_ PACKAGES __________________________________________________________________________ %
	
    %__ INPUT/OUTPUT LANGUAGE _________________________________ %
    \usepackage[USenglish]{babel}
    \usepackage[utf8]{inputenc}
    \usepackage[T1]{fontenc}
    %\usepackage{indentfirst}

    %__ MATH __________________________________________________ %
    \usepackage{amsfonts}
    \usepackage{amssymb}
    \usepackage{amsmath}
    \usepackage{amsthm}
    \usepackage{bbm}
    
    %__ THEORY __________________________________________________ %
    \newtheorem{name}{Printed output}
	\newtheorem{lemma}{Lemma}
	\newtheorem{prop}{Proposition}
	\newtheorem{cor}{Corollary}
	\theoremstyle{definition}
	\newtheorem{definition}{Definition}

    %__ GRAPHS & TABLES________________________________________ %
    \usepackage{graphicx}
    \usepackage{subfig}
    \usepackage{booktabs}
    %\usepackage{multirow}
    \usepackage{array}
    \usepackage{caption}
    %\usepackage{subcaption}
    %\usepackage[flushleft,online,para]{threeparttable}
    
    %\usepackage{parskip}                       % WHAT IS THIS FOR?

    \usepackage{floatrow}
        \floatsetup[table]{style=plaintop}     % LEAVE TABLE CAPTIONS AT THE TOP
    %\usepackage[nolists,nomarkers]{endfloat}             % PUT FIGURES AT THE END OF DOCUMENT; DOESN'T WORK WITH \usepackage{float}

    \usepackage{tabularx}
        \newcolumntype{Z}{>{\centering\arraybackslash}X}
        \newcolumntype{L}{>{\raggedright\arraybackslash}X}

    \usepackage{dcolumn}
        \newcolumntype{d}[1]{D{.}{.}{#1}}

    \usepackage{rotating}                      % for **sideways**tables
    \usepackage{lscape}
    \usepackage{pdflscape}

    %__ BIBLIOGRAPHY __________________________________________ %
    \usepackage[round]{natbib}

    %__ PDF, DISPLAY & PRODUCTIVITY ___________________________ %
    \usepackage{xcolor}
        \definecolor{darkblue}{rgb}{0,0,0.4}
    \usepackage{hyperref}
        \hypersetup{
            colorlinks = true,
            linkcolor = darkblue,
            citecolor = darkblue,
            pdfborder = 0 0 0,
            pdfdisplaydoctitle = true,
            pdfhighlight = /N,
            pdfpagelayout = OneColumn,
            pdfpagemode = UseNone,
            pdfstartview = {FitH},
            pdfauthor = {{AB, CF \& JR}},
            pdftitle = {{DYN}},
            pdfsubject = {{}}
        }

    \usepackage[textsize=footnotesize, colorinlistoftodos, textwidth=4cm, obeyDraft]{todonotes}
    %\usepackage{fixme}
        % commands \fxnote; \fxwarnin; \fxerror; \fxfatal
        % \fxsomething{options}{note}{TEXT}* highlights the TEXT

    \usepackage{geometry}
        \geometry{verbose,tmargin=2.5cm,bmargin=2.5cm,lmargin=2.5cm,rmargin=2.5cm}
    \usepackage{setspace}
        \onehalfspacing

    \usepackage[bottom, multiple]{footmisc}    % keep footnotes at the bottom of the page, and allow for multiple footnotes at one place.

    \usepackage{verbatim}
    \usepackage[normalem]{ulem}     % strikethrough fonts
    \usepackage{mathpazo}

    %\usepackage{syntonly}       % if uncommented, this will prevent latex to produce
    %    \syntaxonly             %   any output. latex will only check for syntax.
    %\usepackage[displaymath,tightpage]{preview}
    %\graphicspath{{//graphs/}}

    %__ APPENDIX _____________________________________________ %
    \usepackage[toc,page]{appendix}

%__ COMMANDS _________________________________________________________________________ %
    \newcommand{\mc}{\multicolumn}
    \newcommand{\lbar}{\underline}
    \newcommand{\ubar}{\overline}


